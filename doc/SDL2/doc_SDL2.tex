\documentclass[ 10pt , a4paper ]{Document}
\usepackage[info]{aMyPack}

\titre{SDL2 - Documentation}
\auteur{Erwan PECHON}

\include{preambule.tex}

\begin{document}
\begin{Document}
\section{Bibliothèque nécessaire}
	Tout fichier contenant de la SDL doit commencer par la ligne :
	\lstinline!#include <SDL2/SDL.h>!.\\
	Cette bibliothèque contient toutes les commandes de bases de la SDL.\\

	Il est aussi possible d'inclure les bibliothèque suivante :\\
	\begin{tabular}{|l@{$\Rightarrow$}p{10cm}|}\hline
		Bibliothèque & utilité\\
		\hline
		\lstinline!#include <SDL2/SDL_ttf.h>!
			& Permet l'installation de police d'écriture\\
			& Permet aussi de simplifier l'affichage de texte.\\
		\lstinline!#include <SDL2/SDL_image.h>!
			& Permet de charger des images de différent type, en tant que surface SDL.
			( BMP, GIF, JPEG, LBM, PCX, PNG, PNM (PPM/PGM/PBM), QOI, TGA, XCF, XPM, and simple SVG )\\
		\hline
	\end{tabular}
\section{Initialisé la SDL}
	Tout programme SDL doit commencé par les lignes suivantes :
\begin{lstlisting}[name="Initialiser"]
if( SDL_Init(Uint32 flags) ){
	printf( "Erreur SDL_Init : %s\n", SDL_GetError() );
	return 1;
}
\end{lstlisting}
	Cela permet d'activer tout les sous-système dont l'on à besoin.
	Pour choisir nos sous-système, il faut donner à l'argument flags, l'une des valeurs suivantes (il est possible dans donnée plusieurs, en les séparant par l'opérateur '\lstinline!|!') :\\
	\begin{tabular}{|c|c|}\hline
		Drapeaux & Description\\
		\hline\hline
		'\lstinline!SDL_INIT_TIMER!' & Initialise le système de gestion du temps\\
		\hline
		'\lstinline!SDL_INIT_AUDIO!' & Initialise le système de gestion de l’audio\\
		\hline
		'\lstinline!SDL_INIT_VIDEO!' & Initialise le système de gestion de rendu\\
		\hline
		'\lstinline!SDL_INIT_JOYSTICK!' & Initialise le système de gestion des joysticks\\
		\hline
		'\lstinline!SDL_INIT_GAMECONTROLLER!' & Initialise le système de gestion des contrôleurs de jeux\\
		\hline
		'\lstinline!SDL_INIT_EVENTS!' & Initialise le système de gestion des évènements\\
		\hline
		'\lstinline!SDL_INIT_EVERYTHING!' & Permet de tout initialiser\\
		\hline
	\end{tabular}
	Après l'initialisation de la SDL, tout erreur doit sauter à la balise '\lstinline!Quit:!', afin de correctement fermer le programme.
	Pour cela, il faudra utilisé la commande '\lstinline!goto(Quit);!'
	à la place de la commande '\lstinline!return(<code erreur>);!'.
\section{Gérer la fenêtre}
	\subsection{Paramétrer la fenêtre}
	Le renderer d'une fenêtre est son contenu. On associe un renderer à une fenêtre, afin de pouvoir calculé l'affichage de la fenêtre, avant de l'afficher, et ainsi éviter des bugs d'affichage.
\begin{lstlisting}[name="Créer une fenêtre et son renderer"]
SDL_Window * window;
SDL_Renderer * renderer;
if( SDL_CreateWindowAndRenderer(int epaisseurW,int hauteurW,Uint32 flagsW,&window,&renderer) ){
	printf( "Erreur SDL_CreateWindowAndRenderer : %s\n", SDL_GetError() );
	goto(Quit);
}
SDL_SetWindowTitle( window , "Titre de la fenêtre" );
\end{lstlisting}
	L'argument flagsW, sert à paramétrer la fenêtre. Il faut lui donner l'une des valeurs suivantes (il est possible dans donnée plusieurs, en les séparant par l'opérateur '\lstinline!|!') :\\
	\begin{tabular}{|c|c|}\hline
		Drapeaux & Description\\
		\hline\hline
		'\lstinline!SDL_WINDOW_FULLSCREEN!' & Crée une fenêtre en plein écran\\
		\hline
		'\lstinline!SDL_WINDOW_FULLSCREEN_DESKTOP!' & Crée une fenêtre en plein écran à la résolution du bureau\\
		\hline
		'\lstinline!SDL_WINDOW_SHOWN!' & Crée une fenêtre visible\\
		\hline
		'\lstinline!SDL_WINDOW_HIDDEN!' & Crée une fenêtre non visible\\
		\hline
		'\lstinline!SDL_WINDOW_BORDERLESS!' & Crée une fenêtre sans bordures\\
		\hline
		'\lstinline!SDL_WINDOW_RESIZABLE!' & Crée une fenêtre redimensionnable\\
		\hline
		'\lstinline!SDL_WINDOW_MINIMIZED!' & Crée une fenêtre minimisée\\
		\hline
		'\lstinline!SDL_WINDOW_MAXIMIZED!' & Crée une fenêtre maximisée\\
		\hline
	\end{tabular}
	Il faut absolument les détruire à la fin du programme avec :
\begin{lstlisting}[name="Détruire une fenêtre et son renderer"]
Quit:
	if( renderer ) SDL_DestroyRenderer( renderer );
	if( window ) SDL_DestroyWindow( window );
	SDL_Quit();
	return 0;
\end{lstlisting}
	\subsection{Gestion des événement}
		\subsubsection{Les événements possibles}
			\begin{tabular}{|p{3cm}|c|c|p{4cm}|}\hline
				Type d’évènements
					& Valeur du champ '\lstinline!<type>!'
					& \begin{minipage}{4cm}
						Champ de '\lstinline!SDL_Event!' correspondant
					\end{minipage}
					& Description
					\\
				\hline\hline
				\multirow{1}*{	\begin{minipage}{5cm}
					Événements\\de l'application
				\end{minipage}	}
					& '\lstinline!SDL_QUIT!'
					& '\lstinline!quit!'
					& Demande de fermeture du programme
					\\
				\hline
				\multirow{2}*{	\begin{minipage}{5cm}
					Évènements\\ de la fenêtre
				\end{minipage}	}
					& '\lstinline!SDL_WINDOWEVENT!'
					& '\lstinline!window!'
					& Changement d’état de la fenêtre
					\\
					& '\lstinline!SDL_SYSWMEVENT!'
					& '\lstinline!syswm!'
					& Évènement dépendant du sytème
					\\
				\hline
				\multirow{4}*{	\begin{minipage}{5cm}
					Événements\\du clavier
				\end{minipage}	}
					& '\lstinline!SDL_KEYDOWN!'
					& '\lstinline!key!'
					& Une touche est pressé
					\\
					& '\lstinline!SDL_KEYUP!'
					& '\lstinline!key!'
					& Une touche est relâchée
					\\
					& '\lstinline!SDL_TEXTEDITING!'
					& '\lstinline!edit!'
					& Édition de texte
					\\
					& '\lstinline!SDL_TEXTINPUT!'
					& '\lstinline!text!'
					& Saisie de texte
					\\
				\hline
				\multirow{4}*{	\begin{minipage}{5cm}
					Événements\\de la souris
				\end{minipage}	}
					& '\lstinline!SDL_SDL_MOUSEMOTION!'
					& '\lstinline!motion!'
					& Déplacement de la souris
					\\
					& '\lstinline!SDL_MOUSEBUTTONDOWN!'
					& '\lstinline!button!'
					& Une touche de la souris est pressée
					\\
					& '\lstinline!SDL_MOUSEBUTTONUP!'
					& '\lstinline!button!'
					& Une touche de la souris est relâchée
					\\
					& '\lstinline!SDL_MOUSEWHEEL!'
					& '\lstinline!wheel!'!'
					& La molette est utilisée
					\\
		\hline
	\end{tabular}
		\subsubsection{Lire un événement}
		Pour lire un événement, nous avons 3 fonctions, que l'ont utilise de la façon suivante :
\begin{lstlisting}[name="Lire un événement"]
SDL_Event event;
if( <fonction>(&event) ){
	switch(event.type){
		case <evenement> :
			<traitement de l'événement>
		case <evenement2> :
			<traitement de l'événement2>
		...
	}
}
\end{lstlisting}
		\paragraph{Attendre qu'un événement arrive}
			La commande '\lstinline!SDL_WaitEvent(&event);!' bloque l’exécution du programme,
			jusqu'à ce qu'un événement arrive, peut importe lequel.\\
			Cette commande renvoi 0 en cas d'erreur et 1 sinon.
		\paragraph{Attendre qu'un événement arrive, pendant un temps limité}
			La commande '\lstinline!SDL_WaitEventTimeout(&event,int time);!' bloque l’exécution du programme,
			jusqu'à :
			\begin{itemize}
				\item ce qu'un événement arrive, peut importe lequel.
				\item ce que \lstinline!<time>! milliseconde ce soit passé.
			\end{itemize}
			Cette commande renvoi 0 en cas d'erreur, ou si aucun événements n'est arrivé avant \lstinline!<time>! milliseconde et 1 sinon.
		\paragraph{Vérifié si un événement est dans la pile}
			La commande '\lstinline!SDL_PollEvent(&event);!' test si il y à des événements dans la file, puis continue.\\
			Cette commande renvoi 1 si elle a lu un événement, et 0 sinon.\\
			Attention, dans ce cas, il faut traité tout les événement de la file d'un seul coup.
			Il faut donc utilisé un \lstinline!while! au lieu d' un \lstinline!if!.
	\subsection{Affichage du contenu}
		Comme dit précédemment, le contenu est préparer dans le renderer, avant de l'afficher à la fenêtre.
		Il y à cependant des moyens de préparer plusieurs contenu à l'avance.
		\paragraph{Les viewports}
			voir '\lstinline!SDL_RenderSetViewport!'
		\paragraph{Les display}
			voir '\lstinline!SDL_SetWindowPosition!'
\section{Dessiner}
	Tout les dessins doivent être fait sur le renderer.
	Il y à plusieurs fonctions de dessins simple dans la SDL :
\begin{lstlisting}[name="Afficher un dessin"]
SDL_PresentRenderer( renderer );
\end{lstlisting}
\begin{lstlisting}[name="Changer la couleur du pinceau"]
if( SDL_SetRendererDrawColor( renderer , int r , int g , int b , int a) ){
	printf( "Erreur, la fonction de dessin à planté : %s\n" , SDL_GetError() );
	goto(Quit);
}
\end{lstlisting}
	Les commandes suivantes utiliseront la dernière couleur de pinceau sélectionner.
\begin{lstlisting}[name="Changer le fond d'écran"]
if( SDL_RendererClear( renderer ) ){
	printf( "Erreur, la fonction de dessin à planté : %s\n" , SDL_GetError() );
	goto(Quit);
}
\end{lstlisting}
	\paragraph{Dessiner des formes géométriques simple}
\end{Document}
\end{document}
